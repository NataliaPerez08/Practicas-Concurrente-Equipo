\documentclass{article}
\usepackage[spanish]{babel}
\usepackage{graphicx}

\title{Cómputo Concurrente 2024-2 \\ Práctica 4 \\  Candados clásicos}

\author{Natalia Abigail Pérez Romero\\
Jonathan Bautista Parra \\ Valeria Reyes Tapia}

\date{\today}

\begin{document}

\maketitle
\section{Introduction}
En esta práctica se implementó un algoritmo de Filtro Modificado y Peterson para solucionar un problema concurrente. 

\section{Problema 1}
El problema 1 esta basado en el problema clásico de los filósofos, en el cual se resolvera con cadados programados por ustedes, siendo el de Peterson y el del Filtro Modificado

\subsection{Cuestionario}

\begin{itemize}
    \item ¿Tu solución cumple con Exclusión mutua? Argumenta porqué.
    \item ¿Tu solución cumple con Deadlock-free? Argumenta porqué.
    \item ¿Tu solución cumple con Libre de Hambruna Starvation-free? Argumenta porqué.
    \item ¿Tu solución cumple con Justicia? Argumenta porqué.
\end{itemize}

\section{Problema 2}
El segundo ejercicio consiste en el de un estacionamiento, en este llegan un número $n$ de carros, dentro del estacionamiento se permite una entrada a un número $m$ tal que $n \geq m$, es decir tiene entrada para $m$ carros, estos se separaran en h niveles o pisos (veanlo con un estacionamiento de plaza comercial, donde hay varios pisos). Los automoviles estaran intentando ingresar de manera constante.

\subsection{Cuestionario}

\begin{itemize}
    \item ¿Tu solución cumple con Exclusión mutua? Argumenta porqué.
    \item ¿Tu solución cumple con Deadlock-free? Argumenta porqué.
    \item ¿Tu solución cumple con Libre de Hambruna Starvation-free? Argumenta porqué.
    \item ¿Tu solución cumple con Justicia? Argumenta porqué.
\end{itemize}

\end{document}