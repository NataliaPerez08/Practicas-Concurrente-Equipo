\documentclass{article}

\title{Title of the Article}
\author{Author Name}
\date{\today}

\begin{document}

\maketitle
\section{Introduction}


\section{Graficas y tablas de valores}

Nuestra contraseña tiene 6 caracteres, por lo que el número de combinaciones posibles es $26^6 = 308 915 776$. En la tabla \ref{tab:my_label} se muestra la aceleración teórica y obtenida en función del número de hilos y porcentaje de código en paralelo.
En este caso, consideramos el porcentaje de código paralelizable como el $cociente=\frac{26^6}{hilos}*10$, dado que usamos este cociente para repartir el trabajo entre los hilos (las posibles combinaciones de las cadenas).

\begin{table}[h]
\centering
\begin{tabular}{|c|c|c|c|}
\hline
\textbf{Número de hilos} & \textbf{Aceleración teórica} & \textbf{Aceleración obtenida} & \textbf{\% código en paralelo} \\
\hline
1 & - & - & 30.89 \% \\
2 & - & - & 15.44 \% \\
27 & - & - & 11.44 \% \\
100 & - & - & 308915776\% \\
\hline
\end{tabular}
\caption{Tabla de aceleración teórica y obtenida en función del número de hilos y porcentaje de código en paralelo.}
\label{tab:my_label}
\end{table}


\section{Cuestionario}
\begin{enumerate}
    \item ¿Cuál fue mi contraseña?
    
    La contraseña es: hlsbrr

    \item ¿Cuántas posibles contraseñas hay?

    Si la contraseña tiene entre 7 y 13 caracteres, por lo que el numero de combinaciones posibles es:
    \[ \Sigma_{i=7}^{i=13} 26^i = 2 147 583 647\]

    Nuestra contraseña es de 6 caracteres, por lo que el numero de combinaciones posibles es:
    \[ 26^6 = 308 915 776\]

    \item ¿La ley de Amdahl siempre se cumple?
    
    No. La ley de Amdahl no siempre se cumple en la práctica. La ley establece que la mejora de la velocidad de un componente de un sistema solo puede mejorar el rendimiento general del sistema hasta un límite determinado. El límite está determinado por la fracción del tiempo que se ejecuta el componente en cuestión.

    \item ¿En qué casos no se cumple?
    
    La introducción de paralelismo puede llevar la necesidad de sincronización entre hilos o procesos, la comunicación entre ellos y la gestión de recursos compartidos. Esto puede contrarrestar parcial o totalmente los beneficios del paralelismo. Problemas con dependencias de datos, donde la salida de una operación es necesaria como entrada para otra, pueden limitar la capacidad de paralelización. 
   
    \item ¿Por qué crees a que se debe esto?
    
    Los sistemas no son perfectamente paralelos: La mayoría de los sistemas tienen algunos componentes que son secuenciales.
    Los componentes no siempre se ejecutan al máximo rendimiento: Los componentes pueden estar inactivos o ejecutarse a una velocidad inferior a su máximo rendimiento por diversas razones, como la falta de recursos o la espera de la entrada del usuario.

    \item ¿Cuál sería la mejora máxima? Es decir, la aceleración teórica máxima
    
    La mejora máxima o aceleración teórica máxima según la Ley de Amdahl se puede calcular utilizando la fórmula proporcionada por la misma ley:
    \[S= \frac{1}{F_s-\frac{F_p}{P}}\]
    Donde:
    \begin{itemize}
        \item S es la mejora máxima
        \item $F_s$ es la fracción del código que se ejecuta de manera secuencial.
        \item $F_p$ es la fracción del código que se puede paralelizar.
        \item P es el número de procesadores
    \end{itemize}
    La mejora máxima se alcanza cuando P tiende a infinito, es decir, cuando se utiliza un número infinito de procesadores:
    \[S = \frac{1}{F_s}\]

    Este resultado muestra que la mejora máxima está inversamente proporcional a la fracción secuencial del código $(F_s)$ Cuanto menor sea $(F_s)$, mayor será la mejora máxima posible. 

    \item Escribe tus conclusiones, además de lo que aprendiste en esta práctica, contratiempos y descubrimientos que hubo durante su realización.
    
    \item ¿Cuál es su rol? 
    
    \item ¿Cuál es mi rol?

    
\end{enumerate}

\end{document}
