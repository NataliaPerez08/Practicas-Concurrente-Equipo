\documentclass{article}

\title{Title of the Article}
\author{Author Name}
\date{\today}

\begin{document}

\maketitle

\section{Introduction}

\section{Cuestionario}

\begin{enumerate}
    \item ¿Cuál fue mi contraseña?
    
    \item ¿Cuántas posibles contraseñas hay?
    Si la contraseña la contrasenna tiene entre 7 y 13 caracteres, por lo que el numero de combinaciones posibles es:
    \[ \Sigma_{i=7}^{i=13} 26^i = 2 147 583 647\]
    Nuestra contraseña es de 6 caracteres, por lo que el numero de combinaciones posibles es:
    \[ 26^6 = 308 915 776\]

    \item ¿La ley de Amdahl siempre se cumple?
    No. La ley de Amdahl no siempre se cumple en la práctica. La ley establece que la mejora de la velocidad de un componente de un sistema solo puede mejorar el rendimiento general del sistema hasta un límite determinado. El límite está determinado por la fracción del tiempo que se ejecuta el componente en cuestión.

    \item ¿En qué casos no se cumple?
    Los sistemas no son perfectamente paralelos: La mayoría de los sistemas tienen algunos componentes que son secuenciales, lo que significa que no pueden ejecutarse al mismo tiempo. Esto significa que la mejora de la velocidad de un componente no siempre puede mejorar el rendimiento general del sistema.
    Los componentes no siempre se ejecutan al máximo rendimiento: Los componentes pueden estar inactivos o ejecutarse a una velocidad inferior a su máximo rendimiento por diversas razones, como la falta de recursos o la espera de la entrada del usuario. Esto significa que la mejora de la velocidad de un componente no siempre puede mejorar el rendimiento general del sistema.
    Los sistemas pueden estar limitados por otros factores: El rendimiento de un sistema puede estar limitado por otros factores, como el ancho de banda de la memoria o la velocidad de la red. Esto significa que la mejora de la velocidad de un componente no siempre puede mejorar el rendimiento general del sistema.

    \item ¿Por qué crees a que se debe esto?
    
    \item ¿Cuál sería la mejora máxima? Es decir, la aceleración teórica máxima
    
    \item Escribe tus conclusiones, además de lo que aprendiste en esta práctica, contratiempos y descubrimientos que hubo durante su realización.
    
    \item ¿Cuál es su rol?
    
    \item ¿Cuál es mi rol?

    
\end{enumerate}

\end{document}
